\usepackage{lipsum}

\title{Example Slides}
\subtitle{Crazyness Distilled}
\author[Thomas Wood]{Thomas Wood \\ \url{https://github.com/edgemaster/beamer-lecture-notes}}
\institute{}
\date{}

\begin{document}

\begin{frame}
  \maketitle
\end{frame}

%% Table of contents
\begin{frame}{Outline}
  \tableofcontents
\end{frame}

\section{Introduction}
\begin{frame}{Course Structure}
  \begin{itemize}
    \item Foo
    \item Bar
    \item Baz
  \end{itemize}
\end{frame}
\lipsum[1]   %% Sample text generator


%% theorem/proof/lemma/corollary/example blocks
\section{Prime Numbers}
\subsection{Definition}
\begin{frame}{What Are Prime Numbers?}
  \begin{definition}
    A \alert{prime number} is a number that has exactly two divisors.
  \end{definition}
  \begin{example}
    \begin{itemize}
      \item 2 is prime (two divisors: 1 and 2).
      \item 3 is prime (two divisors: 1 and 3).
      \item 4 is not prime (\alert{three} divisors: 1, 2, and 4).
    \end{itemize}
  \end{example}
\end{frame}

\section{More sections}
%% Columns
\section{Future}
\begin{frame}
  \frametitle{What’s Still To Do?}
  \begin{columns}
    \column{.5\textwidth}
      \begin{block}{Answered Questions}
        How many primes are there?
      \end{block}
    \column{.5\textwidth}
      \begin{block}{Open Questions}
        Is every even number the sum of two primes?
      \end{block}
  \end{columns}
\end{frame}

\end{document}
